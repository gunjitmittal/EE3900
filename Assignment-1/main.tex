\let\negmedspace\undefined{}
\let\negthickspace\undefined{}
\documentclass[journal,12pt,twocolumn]{IEEEtran}
 \usepackage{gensymb}
 \usepackage{polynom}
\usepackage{amssymb}
\usepackage[cmex10]{amsmath}
\usepackage{amsthm}
 \usepackage{stfloats}
\usepackage{bm} 
 \usepackage{longtable}
 \usepackage{enumitem}
 \usepackage{mathtools}
 \usepackage{tikz}
 \usepackage[breaklinks=true]{hyperref}
\usepackage{listings}
\usepackage{color}                                            
\usepackage{array}                                            
\usepackage{longtable}                                        
\usepackage{calc}                                             
    \usepackage{multirow}                                         
    \usepackage{hhline}                                           
    \usepackage{ifthen}                                           
    \usepackage{lscape} 
	\usepackage{siunitx}
	\usepackage{lstautogobble}

\DeclareMathOperator\erf{erf}    
\DeclareMathOperator*{\Res}{Res}
\DeclareMathOperator*{\equals}{=}
\renewcommand\thesection{\arabic{section}}
\renewcommand\thesubsection{\thesection.\arabic{subsection}}
\renewcommand\thesubsubsection{\thesubsection.\arabic{subsubsection}}
\renewcommand\thesectiondis{\arabic{section}}
\renewcommand\thesubsectiondis{\thesectiondis.\arabic{subsection}}
\renewcommand\thesubsubsectiondis{\thesubsectiondis.\arabic{subsubsection}}
\hyphenation{op-tical net-works semi-conduc-tor}
\def\inputGnumericTable{}                                 %%
\lstset {
	frame=single, 
	breaklines=true,
	columns=fullflexible,
	autogobble=true
}
\begin{document}
\newtheorem{theorem}{Theorem}[section]
\newtheorem{problem}{Problem}
\newtheorem{proposition}{Proposition}[section]
\newtheorem{lemma}{Lemma}[section]
\newtheorem{corollary}[theorem]{Corollary}
\newtheorem{example}{Example}[section]
\newtheorem{definition}[problem]{Definition}
\newcommand{\BEQA}{\begin{eqnarray}}
\newcommand{\EEQA}{\end{eqnarray}}
\newcommand{\define}{\stackrel{\triangle}{=}}
\newcommand*\circled[1]{\tikz[baseline= (char.base)]{
    \node[shape=circle,draw,inner sep=2pt] (char) {#1};}}
\bibliographystyle{IEEEtran}
\providecommand{\mbf}{\mathbf}
\providecommand{\mygtrless}{
  \mathrel{
  \smash{
  \vcenter{
    \offinterlineskip{}
    \ialign{
       \hfil##\hfil\cr % just one centered column
       $0$\cr 
       \noalign{\kern-.3ex}
       $>$\cr 
       \noalign{\kern-.3ex}
       $<$\cr 
       \noalign{\kern-.3ex}
       $1$\cr 
    }% end of the \ialign
  }% end of \vcenter
  }% end of \smash
  \vphantom{>}% pretend it's as high as a >
  }% end of \mathrel
}
\providecommand{\gauss}[2]{\mathcal{N}\ensuremath{\left(#1,#2\right)}}
\providecommand{\pr}[1]{\ensuremath{\Pr\left(#1\right)}}
\providecommand{\qfunc}[1]{\ensuremath{Q\left(#1\right)}}
\providecommand{\sbrak}[1]{\ensuremath{{}\left[#1\right]}}
\providecommand{\lsbrak}[1]{\ensuremath{{}\left[#1\right.]}}
\providecommand{\rsbrak}[1]{\ensuremath{{}\left[#1\right.]}}
\providecommand{\brak}[1]{\ensuremath{\left(#1\right)}}
\providecommand{\lbrak}[1]{\ensuremath{\left(#1\right.)}
\providecommand{\rbrak}[1]{\ensuremath{\left[#1\right.]}}}
\providecommand{\cbrak}[1]{\ensuremath{\left\{#1\right\}}}
\providecommand{\lcbrak}[1]{\ensuremath{\left\{#1\right.}}
\providecommand{\rcbrak}[1]{\ensuremath{\left.#1\right\}}}
\providecommand{\ztrans}{\overset{\mathcal{Z}}{ \rightleftharpoons}}
\theoremstyle{remark}
\newtheorem{rem}{Remark}
\newcommand{\sgn}{\mathop{\mathrm{sgn}}}
\providecommand{\abs}[1]{\left\vert#1\right\vert}
\providecommand{\res}[1]{\Res\displaylimits_{#1}} 
\providecommand{\norm}[1]{\left\lVert#1\right\rVert}
\providecommand{\mtx}[1]{\mathbf{#1}}
\providecommand{\mean}[1]{E\left[ #1 \right]}
\providecommand{\fourier}{\overset{\mathcal{F}}{ \rightleftharpoons}}
\providecommand{\system}{\overset{\mathcal{H}}{ \longleftrightarrow}}
\newcommand{\solution}{\noindent \textbf{Solution: }}
\newcommand{\cosec}{\,\text{cosec}\,}
\newcommand*{\permcomb}[4][0mu]{{{}^{#3}\mkern#1#2_{#4}}}
\newcommand*{\perm}[1][-3mu]{\permcomb[#1]{P}}
\newcommand*{\comb}[1][-1mu]{\permcomb[#1]{C}}
\renewcommand{\thetable}{\arabic{table}} 
\providecommand{\dec}[2]{\ensuremath{\overset{#1}{\underset{#2}{\gtrless}}}}
\newcommand{\myvec}[1]{\ensuremath{\begin{pmatrix}#1\end{pmatrix}}}
\newcommand{\mydet}[1]{\ensuremath{\begin{vmatrix}#1\end{vmatrix}}}
\newcommand{\mymat}[1]{\ensuremath{\begin{bmatrix}#1\end{bmatrix}}}
\numberwithin{equation}{section}
\numberwithin{figure}{section}
\numberwithin{table}{section} 
\makeatletter
\@addtoreset{figure}{problem} 
\makeatother
\let\StandardTheFigure\thefigure{} 
\let\vec\mathbf{}
\def\putbox#1#2#3{\makebox[0in][l]{\makebox[#1][l]{}\raisebox{\baselineskip}[0in][0in]{\raisebox{#2}[0in][0in]{#3}}}}
     \def\rightbox#1{\makebox[0in][r]{#1}}
     \def\centbox#1{\makebox[0in]{#1}}
     \def\topbox#1{\raisebox{-\baselineskip}[0in][0in]{#1}}
     \def\midbox#1{\raisebox{-0.5\baselineskip}[0in][0in]{#1}}
\vspace{3cm}
\title{Digital Signal Processing}
\author{Gunjit Mittal (AI21BTECH11011)}
\maketitle
\tableofcontents 
\renewcommand{\thefigure}{\theenumi}
\renewcommand{\thetable}{\theenumi}
%\renewcommand{\theequation}{\thesection}
\bigskip
\begin{abstract}
This manual provides a simple introduction to digital signal processing.
\end{abstract}
\section{Software Installation}
Run the following commands
\begin{lstlisting}
sudo apt-get update
sudo apt-get install libffi-dev libsndfile1 python3-scipy  python3-numpy python3-matplotlib 
sudo pip install cffi pysoundfile 
\end{lstlisting}
\section{Digital Filter}
\begin{enumerate}[label=\thesection.\arabic*
,ref=\thesection.\theenumi]
\item
\label{prob:input}
Download the sound file from  
\begin{lstlisting}
wget https://github.com/gunjitmittal/EE3900/blob/main/Assignment-1/codes/Sound_Noise.wav
\end{lstlisting}
%\href{http://tlc.iith.ac.in/img/sound/Sound_Noise.wav}{\url{http://tlc.iith.ac.in/img/sound/Sound_Noise.wav}}  
%in the link given below.
%\linebreak
\item
\label{prob:spectrogram}
You will find a spectrogram at \href{https://academo.org/demos/spectrum-analyzer}{\url{https://academo.org/demos/spectrum-analyzer}}. 
%\end{problem}
%%
%
%%\onecolumn
%%\input{./figs/fir}
%\begin{problem}
Upload the sound file that you downloaded in Problem \ref{prob:input} in the spectrogram  and play.  Observe the spectrogram. What do you find?
\\
%
\solution 
By observing spectrogram, it clearly shows that tonal frequency is under 4kHz. And above 4kHz only noise is present.
\item \label{prob:output}
Write the python code for removal of out of band noise and execute the code.
\\
\solution
\lstinputlisting{./codes/noise_rm.py}
\item
The output of the python script in Problem \ref{prob:output} is the audio file Sound\_With\_ReducedNoise.wav. Play the file in the spectrogram in Problem \ref{prob:spectrogram}. What do you observe?
\\
\solution The audio is subdued and the higher frequenices are just blank.
\end{enumerate}
\section{Difference Equation}
\begin{enumerate}[label=\thesection.\arabic*,ref=\thesection.\theenumi]
\item Let
\label{def:xn}
\begin{equation}
x(n) = \cbrak{\underset{\uparrow}{1},2,3,4,2,1}
\end{equation}
Sketch $x(n)$.
\item Let
\begin{multline}
\label{eq:iir_filter}
y(n) + \frac{1}{2}y(n-1) = x(n) + x(n-2), 
\\
 y(n) = 0, n < 0
\end{multline}
Sketch $y(n)$.
\\
\solution The following code yields Fig. \ref{fig:xnyn}.
\begin{lstlisting}
wget https://github.com/gunjitmittal/EE3900/blob/main/Assignment-1/codes/xnyn.py
\end{lstlisting}
\begin{figure}[!ht]
\begin{center}
\includegraphics[width=\columnwidth]{./figs/xnyn}
\end{center}
\caption{figure}{}
\label{fig:xnyn}	
\end{figure}
\item Repeat the above exercise using a C code.
\solution
Download and run the C code for generating y and Python code for plotting y
\begin{lstlisting}
  wget https://github.com/gunjitmittal/EE3900/blob/main/Assignment-1/codes/xnyn.c
  wget https://github.com/gunjitmittal/EE3900/blob/main/Assignment-1/codes/xnyn(1).py
  \end{lstlisting}
  \begin{figure}[!ht]
    \begin{center}
    \includegraphics[width=\columnwidth]{./figs/xnyn(1)}
    \end{center}
    \caption{figure}{}
    \label{fig:xnyn(1)}	
    \end{figure}
\end{enumerate}
\section{$Z$-transform}
\begin{enumerate}[label=\thesection.\arabic*]
\item The $Z$-transform of $x(n)$ is defined as
%
\begin{equation}
\label{eq:z_trans}
X(z)={\mathcal {Z}}\{x(n)\}=\sum _{n=-\infty }^{\infty }x(n)z^{-n}
\end{equation}
%
Show that
\begin{equation}
\label{eq:shift1}
{\mathcal {Z}}\{x(n-1)\} = z^{-1}X(z)
\end{equation}
and find
\begin{equation}
	{\mathcal {Z}}\{x(n-k)\} 
\end{equation}
\solution From \eqref{eq:z_trans},
\begin{align}
{\mathcal {Z}}\{x(n-1)\} &=\sum _{n=-\infty }^{\infty }x(n-1)z^{-n}
\\
&=\sum _{n=-\infty }^{\infty }x(n)z^{-n-1} \\
&=z^{-1}\sum _{n=-\infty }^{\infty }x(n)z^{-n}
\end{align}
resulting in \eqref{eq:shift1}. Similarly, it can be shown that
%
\begin{align}
  {\mathcal {Z}}\{x(n-k)\} &=\sum _{n=-\infty }^{\infty }x(n-k)z^{-n}
  \\
  &=\sum _{n=-\infty }^{\infty }x(n)z^{-n-k} \\
  &=z^{-k}\sum _{n=-\infty }^{\infty }x(n)z^{-n}
  \label{eq:z_trans_shift}
  \end{align}
\item Obtain $X(z)$ for $x(n)$ defined in problem 
	\ref{def:xn}.
\solution \\
\begin{align}
  X(z)&=\sum _{n=0}^{5}x(n)z^{-n}\\
  &= 1 + 2z^{-1} + 3z^{-2} + 4z^{-3} + 2z^{-4} + 1z^{-5}
\end{align}
\item Find
%
\begin{equation}
H(z) = \frac{Y(z)}{X(z)}
\end{equation}
%
from  \eqref{eq:iir_filter} assuming that the $Z$-transform is a linear operation.
\\
\solution We have
\begin{align}
  y(n) + \frac{1}{2}y(n-1) = x(n) + x(n-2)
\end{align} 
Finding the Z-transform 
\begin{align}
  {\mathcal {Z}}\{y(n) + \frac{1}{2}y(n-1)\} &= {\mathcal {Z}}\{x(n)+x(n-2)\}
\end{align}
Because Z-transform is a linear function
\begin{align}
  {\mathcal {Z}}y(n) + \frac{1}{2}{\mathcal {Z}}y(n-1) &= {\mathcal {Z}}x(n)+{\mathcal {Z}}x(n-2)
\end{align}
Using \eqref{eq:z_trans_shift}
\begin{align}
  Y(z) + \frac{1}{2}z^{-1}Y(z) &= X(z)+z^{-2}X(z)
\end{align}
\begin{align}
\implies \frac{Y(z)}{X(z)} &= \frac{1 + z^{-2}}{1 + \frac{1}{2}z^{-1}}
\label{eq:freq_resp}
\end{align}
\item Find the Z transform of 
\begin{equation}
\delta(n)
=
\begin{cases}
1 & n = 0
\\
0 & \text{otherwise}
\end{cases}
\end{equation}
and show that the $Z$-transform of
\begin{equation}
\label{eq:unit_step}
u(n)
=
\begin{cases}
1 & n \ge 0
\\
0 & \text{otherwise}
\end{cases}
\end{equation}
is
\begin{equation}
U(z) = \frac{1}{1-z^{-1}}, \quad \abs{z} > 1
\end{equation}
\solution 
\begin{align}
\delta(n) &\ztrans \sum _{n=-\infty }^{\infty }\delta(n)z^{-n}\\
&= \sum _{n=0}z^{-n} = 1
\end{align}
and from \eqref{eq:unit_step},
\begin{align}
u(n) &\ztrans U(z) = \sum _{n=-\infty }^{\infty }u(n)z^{-n}\\
U(z) &= \sum _{n= 0}^{\infty}z^{-n}\\
&=\frac{1}{1-z^{-1}}, \quad \abs{z} > 1
\end{align}
using the fomula for the sum of an infinite geometric progression.
%
\item Show that 
\begin{equation}
\label{eq:anun}
a^nu(n) \ztrans \frac{1}{1-az^{-1}} \quad \abs{z} > \abs{a}
\end{equation}
\solution
\begin{align}
  a^nu(n) &\ztrans \sum _{n=-\infty }^{\infty }a^n u(n)z^{-n}\\
  &= \sum _{n=0 }^{\infty } (az^{-1})^{-n}\\
  &=\frac{1}{1-az^{-1}}, \quad \abs{z} > \abs{a}
\end{align}
\item 
Let
\begin{equation}
H\brak{e^{\j \omega}} = H\brak{z = e^{\j \omega}}.
\end{equation}
Plot $\abs{H\brak{e^{\j \omega}}}$.  Comment.  $H(e^{\j \omega})$ is
known as the {\em Discret Time Fourier Transform} (DTFT) of $h(n)$.
\\
\solution
\begin{align}
  H\brak{e^{\j \omega}} &= \frac{1 + e^{-2\j\omega}}{1 + \frac12 e^{-\j\omega}} \\
  \implies \abs{H\brak{e^{\j \omega}}} &= \frac{\abs{1 + \cos2\omega - \j\sin2\omega}}{\abs{1 + \frac12 \cos\omega - \frac12 \sin\omega}} \\
  &= \sqrt{\frac{(1 + \cos2\omega)^2 + (\sin2\omega)^2}{(1 + \frac12 \cos\omega)^2 + (\frac12 \sin\omega)^2}} \\
  &= \sqrt{\frac{2 + 2\cos2\omega}{\frac54 + \cos\omega}} \\
  &= \sqrt{\frac{2(2\cos^2\omega)4}{5 + 4\cos\omega} } \\
  &= \frac{4\abs{\cos\omega}}{\sqrt{5 + 4\cos\omega}}
\end{align}
 The following code plots Fig. \ref{fig:dtft}.
\begin{lstlisting}
wget https://github.com/gunjitmittal/EE3900/blob/main/Assignment-1/codes/dtft.py
\end{lstlisting}
The plot is even and has a period of $2\pi $.
% The plot achieves it's global maxima at $\pm(2n+1)\pi$ and global minima at $\pm 2n\pi$
\begin{figure}[!ht]
\centering
\includegraphics[width=\columnwidth]{./figs/dtft}
\caption{$\abs{H\brak{e^{j\omega}}}$}
\label{fig:dtft}
\end{figure}
\item Express $h(n)$ in terms of $H\brak{e^{j \omega}}$.\\
\solution Since $H\brak{e^{j \omega}}$ is the DTFT of $h(n)$
\begin{align}
  &\int_{-\pi}^{\pi} H\brak{e^{j\omega}} e^{jn\omega} d\omega\\
  &=\int_{-\pi}^{\pi} \brak{\sum_{k=-\infty}^{\infty} h(k) e^{-jk\omega}} e^{jn\omega} d\omega\\
  &=\sum_{k=-\infty}^{\infty} h(k) \int_{-\pi}^{\pi}  e^{j(n-k)\omega} d\omega\\
  &=2\pi \sum_{k=-\infty}^{\infty} h(k) \delta(n-k)\\
  &=2\pi h(n)
\end{align}
\begin{equation}
  \therefore h(n) = \frac{1}{2\pi} \int_{-\pi}^{\pi} H\brak{e^{j\omega}} e^{jn\omega} d\omega
\end{equation}
\end{enumerate}
\section{Impulse Response}
\begin{enumerate}[label=\thesection.\arabic*]
  \item Using long division, find
	\begin{align}
		h(n), \quad n < 5
	\end{align}
	for $H(z)$ in \eqref{eq:freq_resp}
	
	\solution 
  \begin{equation}
		H(z) = \frac{1 + z^{-2}}{1 + \frac12 z^{-1}}
	\end{equation}
	Substitute $z^{-1} = x$
	
	\polylongdiv{1+x^2}{1+\frac12 x}
	
	\begin{align}
		&\implies 1 + z^{-2} = \brak{1 + \frac12 z^{-1}}\brak{-4 + 2z^{-1}} + 5 \\
		&\implies H(z) = -4 + 2z^{-1} + \frac{5}{1 + \frac12 z^{-1}}
	\end{align}
	
	On applying the inverse $Z$-transform on both sides of the equation
	\begin{align}
		H(z) &\ztrans h(n) \\
		-4 &\ztrans -4\delta(n) \\
		2z^{-1} &\ztrans 2\delta(n - 1) \\
		\frac{5}{1 + \frac12 z^{-1}} &\ztrans 5\brak{-\frac12}^n u(n) \\
	\end{align}
	
	Therefore,
	\begin{equation}
		h(n) = -4\delta(n) + 2\delta(n - 1) + 5\brak{-\frac12}^n u(n)
	\end{equation}
  \begin{align}
    &h(0) = -4 + 5 = 1\\
    &h(1) = 2 - 2.5 = -0.5\\
    &h(2) = 1.25\\
    &h(3) = -0.625\\
    &h(4) = 0.3125
  \end{align}
\item \label{prob:impulse_resp}
Find an expression for $h(n)$ using $H(z)$, given that 
%in Problem \ref{eq:ztransab} and \eqref{eq:anun}, given that
\begin{equation}
\label{eq:impulse_resp}
h(n) \ztrans H(z)
\end{equation}
and there is a one to one relationship between $h(n)$ and $H(z)$. $h(n)$ is known as the {\em impulse response} of the
system defined by \eqref{eq:iir_filter}.
\\
\solution From \eqref{eq:freq_resp},
\begin{align}
H(z) &= \frac{1}{1 + \frac{1}{2}z^{-1}} + \frac{ z^{-2}}{1 + \frac{1}{2}z^{-1}}
\\
\implies h(n) &= \brak{-\frac{1}{2}}^{n}u(n) + \brak{-\frac{1}{2}}^{n-2}u(n-2)
\end{align}
using \eqref{eq:anun} and \eqref{eq:z_trans_shift}.
\item Sketch $h(n)$. Is it bounded? Justify theoretically.
\\
\solution The following code plots Fig. \ref{fig:hn}.
\begin{lstlisting}
wget https://github.com/gunjitmittal/EE3900/blob/main/Assignment-1/codes/hn.py
\end{lstlisting}
\begin{figure}[!ht]
\centering
\includegraphics[width=\columnwidth]{./figs/hn}
\caption{$h(n)$ as the inverse of $H(z)$}
\label{fig:hn}
\end{figure}
As we can see from the plot $h(n)$ is bounded.
Theoretically,
	\begin{align}
		&\abs{u(n)} &\le 1 \\
		&\abs{\brak{-\frac12}^n} &\le 1 \\
		\implies &\abs{\brak{-\frac12}^n u(n)} &\le 1
	\end{align}
	
	Similarly,
	\begin{align}
		&\abs{\brak{-\frac12}^{n-2} u(n-2)} &\le 1 \\
		\implies &h(n) &\le 2
	\end{align}
	
	Therefore $h(n)$ is bounded.
  \item Is it convergent? Justify using ratio test.\\
  \solution Using the ratio test for convergence
	\begin{align}
		\lim_{n \to \infty} \abs{\frac{h(n+1)}{h(n)}} &= \lim_{n \to \infty} \abs{\frac{\brak{-\frac12}^{n-1} \brak{\frac14 + 1}}{\brak{-\frac12}^{n-2} \brak{\frac14 + 1}}} \\
		&= \lim_{n \to \infty} \abs{-\frac12} \\
		&= \frac{1}{2} < 1
	\end{align}
	
	Therefore, $h(n)$ is convergent.
\item The system with $h(n)$ is defined to be stable if
\begin{equation}
\sum_{n=-\infty}^{\infty}h(n) < \infty
\end{equation}
Is the system defined by \eqref{eq:iir_filter} stable for the impulse response in \eqref{eq:impulse_resp}?\\
\solution
	\begin{multline}
		\sum_{n=-\infty}^{\infty}h(n) = \sum_{n=-\infty}^{\infty} \brak{-\frac12}^n u(n) \\
		+ \sum_{n=-\infty}^{\infty} \brak{-\frac12}^{n-2} u(n-2)
	\end{multline}
	\begin{align}
		\sum_{n=-\infty}^{\infty}h(n) = \sum_{n=0}^{\infty}\brak{-\frac12}^n + \sum_{n=2}^{\infty}\brak{-\frac12}^{n-2}
	\end{align}
	
	These are both sums of infinite geometric progressions with first terms $1$ and common ratios $-\frac12$
	\begin{align}
		\sum_{n=-\infty}^{\infty}h(n) &= \frac{1}{1 - \brak{-\frac12}} + \frac{1}{1 - \brak{-\frac12}} \\
		&= \frac{4}{3} < \infty
	\end{align}
	Therefore, the system is stable. 
  \item Verify the above result using a Python code. 
	
	\solution The stability has been verified in the following code
  \begin{lstlisting}
    wget https://github.com/gunjitmittal/EE3900/blob/main/Assignment-1/codes/5_6.py
    \end{lstlisting}
\item 
Compute and sketch $h(n)$ using 
\begin{equation}
\label{eq:iir_filter_h}
h(n) + \frac{1}{2}h(n-1) = \delta(n) + \delta(n-2), 
\end{equation}
%
This is the definition of $h(n)$.
\\
\solution 
	\begin{equation}
		h(0) = 1
	\end{equation}
	
	Now, for $n = 1$,
	\begin{align}
		h(1) + \frac12 h(0) &= \delta(1) + \delta(-1) = 0 \\
		\implies h(1) &= - \frac{1}{2} h(0) = -\frac{1}{2}
	\end{align}
	
	For $n = 2$,
	\begin{align}
		h(2) + \frac12 h(1) &= \delta(2) + \delta(0) = 1 \\
		\implies h(2) &= 1 - \frac{1}{2} h(1) = \frac{5}{4}
	\end{align}
	
	For $n > 2$, the right hand side of the equation is always zero. Thus,
	\begin{align}
		h(n) &= -\frac{1}{2} h(n-1) \qquad n > 2 \\
		h(3) &= \frac{5}{4} \brak{-\frac12} \\
		h(4) &= \frac{5}{4} \brak{-\frac12}^2 \\
		&~\vdots \\
		h(n) &= \frac{5}{4} \brak{-\frac12}^{n-2}
	\end{align}
	
	Therefore,
	\begin{align}
		h(n) = 
		\begin{cases}
			1 & n = 0 \\
			-\dfrac{1}{2} & n = 1 \\
			\dfrac{5}{4} \brak{-\dfrac12}^{n-2} & n \ge 2
		\end{cases}
	\end{align}
	
	Thus, it is bounded and convergent to $0$
	\begin{equation}
		\lim_{n \to \infty} h(n) = 0
	\end{equation}
   The following code plots Fig. \ref{fig:hndef}. Note that this is the same as Fig. 
\ref{fig:hn}. 
\begin{lstlisting}
wget https://github.com/gunjitmittal/EE3900/blob/main/Assignment-1/codes/hndef.py
\end{lstlisting}
\begin{figure}[!ht]
\centering
\includegraphics[width=\columnwidth]{./figs/hndef}
\caption{$h(n)$ from the definition}
\label{fig:hndef}
\end{figure}
%
\item Compute 
%
\begin{equation}
\label{eq:convolution}
y(n) = x(n)*h(n) = \sum_{k=-\infty}^{\infty}x(k)h(n-k)
\end{equation}
%
Comment. The operation in \eqref{eq:convolution} is known as
{\em convolution}.
%
\\
\solution The following code plots Fig. \ref{fig:ynconv}. Note that this is the same as 
$y(n)$ in  Fig. 
\ref{fig:xnyn}. 
%
\begin{lstlisting}
wget https://github.com/gunjitmittal/EE3900/blob/main/Assignment-1/codes/ynconv.py
\end{lstlisting}
\begin{figure}[!ht]
\centering
\includegraphics[width=\columnwidth]{./figs/ynconv}
\caption{$y(n)$ from the definition of convolution}
\label{fig:ynconv}
\end{figure}
\item Express the above convolution using a Toeplitz matrix.
	
	\solution Let 
	\begin{align}
		\vec{x} = \myvec{1 \\ 2 \\ 3 \\ 4 \\ 2 \\ 1} \qquad
		\vec{h} = \myvec{1 \\ -0.5 \\ 1.25 \\ -0.62 \\ 0.31 \\ -0.16}
	\end{align}
	
	Their convolution is given by the product of the following Toeplitz matrix $\vec{T}$
	\begin{align}
		\myvec{
			1 & 0 & 0 & 0 & 0 & 0 \\
			-0.5 & 1 & 0 & 0 & 0 & 0 \\
			1.25 & -0.5 & 1 & 0 & 0 & 0 \\
			-0.62 & 1.25 & -0.5 & 1 & 0 & 0 \\
			0.31 & -0.62 & 1.25 & -0.5 & 1 & 0 \\
			-0.16 & 0.31 & -0.62 & 1.25 & -0.5 & 1 \\
			0 & -0.16 & 0.31 & -0.62 & 1.25 & -0.5 \\
			0 & 0 & -0.16 & 0.31 & -0.62 & 1.25 \\
			0 & 0 & 0 & -0.16 & 0.31 & -0.62 \\
			0 & 0 & 0 & 0 & -0.16 & 0.31 \\
			0 & 0 & 0 & 0 & 0 & -0.16 \\
		} 
	\end{align}
	and $\vec{x}$
	
	\begin{align}
		&\vec{y} = \vec{x} \circledast \vec{h} = \vec{Tx} = \myvec{1 \\ 1.5 \\ 3.25 \\ 4.38 \\ 2.81 \\ 3.59 \\ 0.12 \\ 0.78 \\ -0.62 \\ 0 \\ -0.16}
	\end{align}
	
	Download the following Python code for computing the convolution by using a Toeplitz matrix and plotting Fig. \ref{fig-5.9}
	\begin{lstlisting}
		wget https://github.com/gunjitmittal/EE3900/blob/main/Assignment-1/codes/5_9.py
	\end{lstlisting}
	
	Run the Python code by executing
	\begin{lstlisting}
		python 5.9.py
	\end{lstlisting}

	\begin{figure}[!ht]
		\centering
		\includegraphics[width=\columnwidth]{./figs/5.9.pdf}
		\caption{Plot of the convolution of $x(n)$ and $h(n)$}
		\label{fig-5.9}	
	\end{figure}
\item Show that
\begin{equation}
y(n) =  \sum_{k=-\infty}^{\infty}x(n-k)h(k)
\end{equation}
\end{enumerate}
\solution
From \eqref{eq:convolution}
\begin{align}
  y(n) = \sum_{k=-\infty}^\infty x(k)h(n-k)
\end{align}
Substituting $k$ with $n-k$
\begin{align}
~~~~~~~~= \sum_{n-k=-\infty}^\infty x(n-k)h(k)
\end{align}
as $n$ remains constant, we can rewrite it as
\begin{align}
  ~~~~~~= \sum_{k=-\infty}^\infty x(n-k)h(k)
\end{align}
\section{DFT and FFT}
\begin{enumerate}[label=\thesection.\arabic*]
\item
Compute
\begin{equation}
X(k) \define \sum _{n=0}^{N-1}x(n) e^{-\j2\pi kn/N}, \quad k = 0,1,\dots, N-1
\end{equation}
and $H(k)$ using $h(n)$.\\
\solution 
\begin{figure}[!ht]
  \centering
  \includegraphics[width=\columnwidth]{./figs/X-H(n).pdf}
  \caption{$X(n) and H(n)$}
  \label{fig:X-H(n)}
  \end{figure}
  The following code plots Fig. \ref{fig:X-H(n)}.
%
\begin{lstlisting}
wget https://github.com/gunjitmittal/EE3900/blob/main/Assignment-1/codes/6_1.py
\end{lstlisting}
\item Compute 
\begin{equation}
Y(k) = X(k)H(k)
\label{eq:fp}
\end{equation}
\solution 
\begin{figure}[!ht]
  \centering
  \includegraphics[width=\columnwidth]{./figs/Y(n).pdf}
  \caption{$Y(n)$}
  \label{fig:Y(n)}
  \end{figure}
  The following code plots Fig. \ref{fig:Y(n)}.
%
\begin{lstlisting}
wget https://github.com/gunjitmittal/EE3900/blob/main/Assignment-1/codes/6_2.py
\end{lstlisting}
\item Compute
\begin{equation}
  y\brak{n}={\frac {1}{N}}\sum _{k=0}^{N-1}Y\brak{k}\cdot e^{\j 2\pi kn/N},\quad n = 0,1,\dots, N-1
  \label{eq:inv-ft}
 \end{equation}
 \\
 \solution The following code plots Fig. \ref{fig:yndft}. Note that this is the same as 
 $y(n)$ in  Fig. 
 \ref{fig:xnyn}. 
 %
 \begin{lstlisting}
 wget https://github.com/gunjitmittal/EE3900/blob/main/Assignment-1/codes/yndft.py
 \end{lstlisting}
 \begin{figure}[!ht]
 \centering
 \includegraphics[width=\columnwidth]{./figs/yndft}
 \caption{$y(n)$ from the IDFT}
 \label{fig:yndft}
 \end{figure}
 \item Repeat the previous exercise by computing $X(k), H(k)$ and $y(n)$ through FFT and 
 IFFT.\\
 \solution
 \begin{figure}[!ht]
  \centering
  \includegraphics[width=\columnwidth]{./figs/6_4.pdf}
  \caption{$y(n)$ using FFT and IFFT}
  \label{fig:y(n)_FFT}
  \end{figure}
  The following code plots Fig. \ref{fig:y(n)_FFT}.
%
\begin{lstlisting}
wget https://github.com/gunjitmittal/EE3900/blob/main/Assignment-1/codes/6_4.py
\end{lstlisting}
 \end{enumerate}
 %
 \section{FFT}
% \subsection{Definitions}
\begin{enumerate}[label=\arabic*.,ref=\thesection.\theenumi]
\numberwithin{equation}{section}
    \item The DFT of $x(n)$ is given by
    \begin{align}
        X(k) \triangleq \sum_{n=0}^{N-1} x(n) e^{-j 2 \pi k n / N}, \quad k=0,1, \ldots, N-1
    \end{align}
\item Let 
	\begin{align}
W_{N} = e^{-j2\pi/N} 
	\end{align}
		Then the $N$-point {\em DFT matrix} is defined as 
	\begin{align}
		\vec{F}_{N} = \sbrak{W_{N}^{mn}}, \quad 0 \le m,n \le N-1 
	\end{align}
	where $W_{N}^{mn}$ are the elements of $\vec{F}_{N}$.
\item Let 
	\begin{align}
		\vec{I}_4 = \myvec{\vec{e}_4^{1} &\vec{e}_4^{2} &\vec{e}_4^{3} &\vec{e}_4^{4} }
	\end{align}
		be the $4\times 4$ identity matrix.  Then the 4 point {\em DFT permutation matrix} is defined as 
	\begin{align}
		\vec{P}_4 = \myvec{\vec{e}_4^{1} &\vec{e}_4^{3} &\vec{e}_4^{2} &\vec{e}_4^{4} }
	\end{align}
\item The 4 point {\em DFT diagonal matrix} is defined as 
	\begin{align}
		\vec{D}_4 = diag\myvec{W_{8}^{0} & W_{8}^{1} & W_{8}^{2} & W_{8}^{3}}
	\end{align}
\item Show that 
\begin{equation}
    W_{N}^{2}=W_{N/2}
\end{equation}
\solution
\begin{align}
     W_{N} &= e^{-j2\pi/N} \\
     W_{N}^{2} &= e^{-j2 \pi \times 2/N} \\ 
     &=e^{-j2 \pi/(N/2)}\\
     &=W_{N/2}
\end{align}
%    \item Find $\vec{P}_6$.
%    \item Find $\vec{D}_3$.
    \item Show that 
\begin{equation}
	\vec{F}_{4}=
\begin{bmatrix}
	\vec{I}_{2} & \vec{D}_{2} \\
\vec{I}_{2} & -\vec{D}_{2}
\end{bmatrix}
\begin{bmatrix}
\vec{F}_{2} & 0 \\
0 & \vec{F}_{2}
\end{bmatrix}
\vec{P}_{4}
\end{equation}
\solution
	\begin{align}
		&\mymat{\vec{I}_2 & \vec{D}_2 \\ \vec{I}_2 & -\vec{D}_2} \mymat{\vec{F}_2 & 0 \\ 0 & \vec{F}_2} \\
		= &\mymat{\vec{F}_2 & \vec{D}_2\vec{F}_2 \\ \vec{F}_2 & -\vec{D}_2\vec{F}_2}  \\
		= &\mymat{\myvec{1&1\\1&-1} & \myvec{1&0\\0&-\j}\myvec{1&1\\1&-1} \\ \myvec{1&1\\1&-1} & - \myvec{1&0\\0&-\j}\myvec{1&1\\1&-1}} \\
		= &\mymat{1 & 1 & 1 & 1 \\ 1 & -1 & -\j & \j \\ 1 & 1 & -1 & -1 \\ 1 & -1 & \j & -\j}
	\end{align}		
	because $W_2^0 = 1$ and $W_2^1 = e^{-\j\pi} = -1$
	Now
	\begin{align}
		&\mymat{\vec{I}_2 & \vec{D}_2 \\ \vec{I}_2 & -\vec{D}_2} \mymat{\vec{F}_2 & 0 \\ 0 & \vec{F}_2} \vec{P}_4 \\
		= &\mymat{1 & 1 & 1 & 1 \\ 1 & -1 & -\j & \j \\ 1 & 1 & -1 & -1 \\ 1 & -1 & \j & -\j} \mymat{1 & 0 & 0 & 0 \\ 0 & 0 & 1 & 0 \\ 0 & 1 & 0 & 0 \\ 0 & 0 & 0 & 1} \\
		= &\mymat{1 & 1 & 1 & 1 \\ 1 & -\j & -1 & \j \\ 1 & -1 & 1 & -1 \\ 1 & \j & -1 & -\j} \\
		= &\mymat{W_4^0 & W_4^0 & W_4^0 & W_4^0 \\ W_4^0 & W_4^1 & W_4^2 & W_4^3 \\ W_4^0 & W_4^2 & W_4^4 & W_4^6 \\ W_4^0 & W_4^3 & W_4^6 & W_4^9} \\
		= &~\vec{F}_4
	\end{align}
	because
	\begin{align}
		W_4^0 &= 1 \\
		W_4^1 &= e^{-\j\frac{\pi}{2}} = -\j \\
		W_4^2 &= e^{-\j\pi} = -1 \\
		W_4^3 &= e^{-\j\frac{3\pi}{2}} = \j \\
		W_4^n &= W_4^{n - 4} &&\forall n \ge 4
	\end{align}	 	
\item Show that 
\begin{equation}
\vec{F}_{N}=
\begin{bmatrix}
\vec{I}_{N/2} & \vec{D}_{N/2} \\
\vec{I}_{N/2} & -\vec{D}_{N/2}
\end{bmatrix}
\begin{bmatrix}
\vec{F}_{N/2} & 0 \\
0 & \vec{F}_{N/2}
\end{bmatrix}
\vec{P}_{N}
\end{equation}
\solution
	\begin{align}
		&\mymat{\vec{I}_{N/2} & \vec{D}_{N/2} \\ \vec{I}_{N/2} & -\vec{D}_{N/2}} \mymat{\vec{F}_{N/2} & 0 \\ 0 & \vec{F}_{N/2}}  \\
		= &\mymat{\vec{F}_{N/2} & \vec{D}_{N/2}\vec{F}_{N/2} \\ \vec{F}_{N/2} & -\vec{D}_{N/2}\vec{F}_{N/2}}
	\end{align}
	
	Now
	\begin{align}
		&\vec{D}_{N/2}\vec{F}_{N/2} \\
		&= \mymat{W_N^0 & \cdots & 0 \\ \vdots & \ddots & \vdots \\ 0 & \cdots & W_N^{N/2-1}} \mymat{W_{N/2}^0 & \cdots & W_{N/2}^0 \\ \vdots & \ddots & \vdots \\ W_{N/2}^0 & \cdots & W_{N/2}^{(N/2 - 1)^2}} \\
		&= \mymat{W_N^0 W_{N/2}^0 & \cdots & W_N^0 W_{N/2}^0 \\ \vdots & \ddots & \vdots \\ W_N^{N/2-1} W_{N/2}^0 & \cdots & W_N^{N/2-1} W_{N/2}^{(N/2 - 1)^2}} 
	\end{align}
	
	Thus
	\begin{align}
		\brak{\vec{D}_{N/2}\vec{F}_{N/2}}_{ij} &= W_N^i W_{N/2}^{ij} \\
		&= W_N^i W_N^{2ij} \\
		&= W_N^{i(2j + 1)}
	\end{align}
	where $i, j = 0, \ldots, N/2 - 1$
	
	Therefore, $\vec{D}_{N/2}\vec{F}_{N/2}$ forms the first $N/2$ rows of the odd-indexed columns of $\vec{F}_N$
	\begin{align}
		W_N^{(i + N/2)(2j + 1)} &= \exp\brak{-\j \frac{2\pi}{N} (2j + 1)\brak{i + \frac{N}{2}}} \\
		&= \exp\brak{-\j \brak{\frac{2\pi}{N} (2j + 1)i + (2j + 1)\pi}} \\
		&= -\exp\brak{-\j \frac{2\pi}{N} (2j + 1)i} \\
		&= -W_N^{i(2j + 1)}
	\end{align}
	
	Thus, the remaining $N/2$ rows will be the negatives of the first $N/2$ rows
	\begin{align}
		\brak{\vec{F}_{N/2}}_{ij} &= W_{N/2}^{ij} \\
		&= W_N^{i(2j)}
	\end{align}
	where $i, j = 0, \ldots, N/2 - 1$
	
	Therefore, $\vec{F}_{N/2}$ forms the first $N/2$ rows of the even-indexed columns of $\vec{F}_N$
	\begin{align}
		W_N^{(i + N/2)(2j)} &= \exp\brak{-\j \frac{2\pi}{N} (2j)\brak{i + \frac{N}{2}}} \\
		&= \exp\brak{-\j \brak{\frac{2\pi}{N} (2j)i + (2j)\pi}} \\
		&= \exp\brak{-\j \frac{2\pi}{N} (2j)i} \\
		&= W_N^{i(2j)}
	\end{align}
	Thus, the remaining $N/2$ rows will be the same as the first $N/2$ rows
	
	Therefore
	\begin{align}
		\mymat{\vec{F}_{N/2} & \vec{D}_{N/2}\vec{F}_{N/2} \\ \vec{F}_{N/2} & -\vec{D}_{N/2}\vec{F}_{N/2}} = \vec{F}_N \vec{P}_N
	\end{align}
	where 
	\begin{align}
		\vec{P}_N = \myvec{\vec{e}_N^1 & \vec{e}_N^3 & \cdots & \vec{e}_N^{N-1} & \vec{e}_N^2 & \vec{e}_N^4 & \cdots & \vec{e}_N^N}
	\end{align}
	
	Hence
	\begin{align}
		\mymat{\vec{F}_{N/2} & \vec{D}_{N/2}\vec{F}_{N/2} \\ \vec{F}_{N/2} & -\vec{D}_{N/2}\vec{F}_{N/2}} \vec{P}_N = \vec{F}_N \vec{P}_N^2 = \vec{F}_N \\
		\therefore \vec{F}_N = \mymat{\vec{I}_{N/2} & \vec{D}_{N/2} \\ \vec{I}_{N/2} & -\vec{D}_{N/2}} \mymat{\vec{F}_{N/2} & 0 \\ 0 & \vec{F}_{N/2}} \vec{P}_N
	\end{align}
	for even $N$

\item Find 
    \begin{align}
	     \vec{P}_4 \vec{x}
    \end{align}
	\solution Let $\vec{x} = \myvec{x(0) & x(1) & x(2) & x(3)}^\top$
    \begin{align}
		\vec{P}_4 \vec{x} &= \mymat{1 & 0 & 0 & 0 \\ 0 & 0 & 1 & 0 \\ 0 & 1 & 0 & 0 \\ 0 & 0 & 0 & 1} \mymat{x(0) \\ x(1) \\ x(2) \\ x(3)} \\
		&= \mymat{x(0) \\ x(2) \\ x(1) \\ x(3)}
    \end{align}
\item Show that 
    \begin{align}
	    \vec{X} = \vec{F}_N \vec{x}
	    \label{eq:dft-mat-def}
    \end{align}
		where $\vec{x}, \vec{X}$ are the vector representations of $x(n), X(k)$ respectively.
		\solution
	\begin{align}
        X(k) &= \sum_{n=0}^{N-1} x(n) e^{-j 2 \pi k n / N}, \quad k=0,1, \ldots, N-1 \\
        \implies \vec{X} &= \mymat{\sum_{n=0}^{N-1} x(n) e^{-j 2 \pi  n (0) / N} \\ \vdots \\ \sum_{n=0}^{N-1} x(n) e^{-j 2 \pi  n (N-1) / N}} \\
        &= \mymat{x(0) + \cdots + x(N-1) \\ \vdots \\ x(0) + \cdots + x(N-1) e^{-j 2 \pi (N-1)^2 / N}} 
    \end{align}
    \begin{align}
    		\vec{X} &= x(0) \mymat{1 \\ \vdots \\ 1} + \cdots + x(N-1)\mymat{1 \\ \vdots \\ e^{-j 2 \pi (N-1)^2 / N}} \\
    		&= \mymat{1 & \cdots & 1 \\ \vdots & \ddots & \vdots \\ 1 & \cdots & e^{-j 2 \pi (N-1)^2 / N}} \mymat{x(0) \\ \vdots \\ x(N-1)} \\
    		&= \vec{F}_N \vec{x}
    \end{align}
\item Derive the following Step-by-step visualisation  of
8-point FFTs into 4-point FFTs and so on
\begin{equation}
\begin{bmatrix}
X(0) \\ 
X(1) \\ 
X(2) \\ 
X(3)
\end{bmatrix}
=
\begin{bmatrix}
X_{1}(0) \\ 
X_{1}(1)\\ 
X_{1}(2)\\
X_{1}(3)\\
\end{bmatrix}
+
\begin{bmatrix}
W^{0}_{8} & 0 & 0 & 0\\
0 & W^{1}_{8} & 0 & 0\\
0 & 0 & W^{2}_{8} & 0\\
0 & 0 & 0 & W^{3}_{8}
\end{bmatrix}
\begin{bmatrix}
X_{2}(0) \\ 
X_{2}(1) \\ 
X_{2}(2) \\
X_{2}(3)
\end{bmatrix}
\end{equation}
\begin{equation}
\begin{bmatrix}
X(4) \\ 
X(5) \\ 
X(6) \\ 
X(7)
\end{bmatrix}
=
\begin{bmatrix}
X_{1}(0) \\ 
X_{1}(1)\\ 
X_{1}(2)\\
X_{1}(3)\\
\end{bmatrix}
-
\begin{bmatrix}
W^{0}_{8} & 0 & 0 & 0\\
0 & W^{1}_{8} & 0 & 0\\
0 & 0 & W^{2}_{8} & 0\\
0 & 0 & 0 & W^{3}_{8}
\end{bmatrix}
\begin{bmatrix}
X_{2}(0) \\ 
X_{2}(1) \\ 
X_{2}(2) \\
X_{2}(3)
\end{bmatrix}
\end{equation}
4-point FFTs into 2-point FFTs
\begin{equation}
\begin{bmatrix}
X_{1}(0) \\ 
X_{1}(1)\\ 
\end{bmatrix}
=
\begin{bmatrix}
X_{3}(0) \\ 
X_{3}(1)\\ 
\end{bmatrix}
+
\begin{bmatrix}
W^{0}_{4} & 0\\
0 & W^{1}_{4}
\end{bmatrix}
\begin{bmatrix}
X_{4}(0) \\ 
X_{4}(1) \\ 
\end{bmatrix}
\end{equation}
\begin{equation}
\begin{bmatrix}
X_{1}(2) \\ 
X_{1}(3)\\ 
\end{bmatrix}
=
\begin{bmatrix}
X_{3}(0) \\ 
X_{3}(1)\\ 
\end{bmatrix}
-
\begin{bmatrix}
W^{0}_{4} & 0\\
0 & W^{1}_{4}
\end{bmatrix}
\begin{bmatrix}
X_{4}(0) \\ 
X_{4}(1) \\ 
\end{bmatrix}
\end{equation}
\begin{equation}
\begin{bmatrix}
X_{2}(0) \\ 
X_{2}(1)\\ 
\end{bmatrix}
=
\begin{bmatrix}
X_{5}(0) \\ 
X_{5}(1)\\ 
\end{bmatrix}
+
\begin{bmatrix}
W^{0}_{4} & 0\\
0 & W^{1}_{4}
\end{bmatrix}
\begin{bmatrix}
X_{6}(0) \\ 
X_{6}(1) \\ 
\end{bmatrix}
\end{equation}
\begin{equation}
\begin{bmatrix}
X_{2}(2) \\ 
X_{2}(3)\\ 
\end{bmatrix}
=
\begin{bmatrix}
X_{5}(0) \\ 
X_{5}(1)\\ 
\end{bmatrix}
-
\begin{bmatrix}
W^{0}_{4} & 0\\
0 & W^{1}_{4}
\end{bmatrix}
\begin{bmatrix}
X_{6}(0) \\ 
X_{6}(1) \\ 
\end{bmatrix}
\end{equation}
\begin{equation}
P_{8}
\begin{bmatrix}
x(0) \\ 
x(1) \\ 
x(2) \\ 
x(3) \\ 
x(4) \\ 
x(5) \\
x(6) \\
x(7)
\end{bmatrix}
 = 
\begin{bmatrix}
x(0) \\ 
x(2) \\ 
x(4) \\ 
x(6) \\
x(1) \\ 
x(3) \\ 
x(5) \\
x(7)
\end{bmatrix}
\end{equation}
\begin{equation}
P_{4}
\begin{bmatrix}
x(0) \\ 
x(2) \\ 
x(4) \\ 
x(6) \\
\end{bmatrix}
 = 
\begin{bmatrix}
x(0) \\ 
x(4) \\ 
x(2) \\
x(6)
\end{bmatrix}
\end{equation}
\begin{equation}
P_{4}
\begin{bmatrix}
x(1) \\ 
x(3) \\ 
x(5) \\
x(7)
\end{bmatrix}
 = 
\begin{bmatrix}
x(1) \\ 
x(5) \\ 
x(3) \\ 
x(7) \\
\end{bmatrix}
\end{equation}
Therefore,
\begin{equation}
\begin{bmatrix}
X_{3}(0) \\ 
X_{3}(1)\\ 
\end{bmatrix}
= F_{2}
\begin{bmatrix}
x(0) \\ 
x(4) \\ 
\end{bmatrix}
\end{equation}
\begin{equation}
\begin{bmatrix}
X_{4}(0) \\ 
X_{4}(1)\\ 
\end{bmatrix}
= F_{2}
\begin{bmatrix}
x(2) \\ 
x(6) \\ 
\end{bmatrix}
\end{equation}
\begin{equation}
\begin{bmatrix}
X_{5}(0) \\ 
X_{5}(1)\\ 
\end{bmatrix}
= F_{2}
\begin{bmatrix}
x(1) \\ 
x(5) \\ 
\end{bmatrix}
\end{equation}
\begin{equation}
\begin{bmatrix}
X_{6}(0) \\ 
X_{6}(1)\\ 
\end{bmatrix}
= F_{2}
\begin{bmatrix}
x(3) \\ 
x(7) \\ 
\end{bmatrix}
\end{equation}
\solution 
	\begin{align}
        X(k) &= \sum_{n=0}^7 x(n) e^{-j 2 \pi k n / 8}, \quad k=0, \ldots, 7\\
        &= \sum_{n=0}^7 x(n) W_8^{kn} \\
        &= \sum_{n \text{ is even}}x(n) W_8^{kn} + \sum_{n \text{ is odd}}x(n) W_8^{kn} \\
        &= \sum_{m=0}^3 x(2m) W_8^{2km} + \sum_{m=0}^3 x(2m+1) W_8^{2km + k}
    \end{align}	
    
    Now substitute $W_8^2 = W_4$
    \begin{align}
    		X(k) = \sum_{m=0}^3 x(2m) W_4^{km} + W_8^k \sum_{m=0}^3 x(2m+1) W_4^{km}
    \end{align}
    
    Consider
    \begin{align}
    		x_1(n) = \cbrak{x(0), x(2), x(4), x(6)} \\
    		x_2(n) = \cbrak{x(1), x(3), x(5), x(7)}
    \end{align}
    
    Thus
    \begin{align}
    		X(k) = X_1(k) + W_8^k X_2(k) \qquad k = 0,\ldots,7
    \end{align}
    
    Now, $X_1(k)$ and $X_2(k)$ are $4$-point DFTs which means they are periodic with period $4$
    \begin{align}
    		X(k+4) &= X_1(k+4) + W_8^{k+4} X_2(k+4) \\
    		&= X_1(k) + e^{-\j 2\pi(k+4)/8} X_2(k) \\
    		&= X_1(k) + e^{-\j (2\pi k/8 + \pi)} X_2(k) \\
    		&= X_1(k) - e^{-\j 2\pi k/8 } X_2(k) \\
    		&= X_1(k) - W_8^k X_2(k)
    \end{align}
    
    Therefore, for $k=0,1,2,3$
    \begin{align}
    		X(k) = X_1(k) + W_8^k X_2(k)  \\
    		X(k+4) = X_1(k) - W_8^k X_2(k) 
    \end{align}
    
    which is the same as
    \begin{equation}
\begin{bmatrix}
X(0) \\ 
X(1) \\ 
X(2) \\ 
X(3)
\end{bmatrix}
=
\begin{bmatrix}
X_{1}(0) \\ 
X_{1}(1)\\ 
X_{1}(2)\\
X_{1}(3)\\
\end{bmatrix}
+
\begin{bmatrix}
W^{0}_{8} & 0 & 0 & 0\\
0 & W^{1}_{8} & 0 & 0\\
0 & 0 & W^{2}_{8} & 0\\
0 & 0 & 0 & W^{3}_{8}
\end{bmatrix}
\begin{bmatrix}
X_{2}(0) \\ 
X_{2}(1) \\ 
X_{2}(2) \\
X_{2}(3)
\end{bmatrix}
\end{equation}
\begin{equation}
\begin{bmatrix}
X(4) \\ 
X(5) \\ 
X(6) \\ 
X(7)
\end{bmatrix}
=
\begin{bmatrix}
X_{1}(0) \\ 
X_{1}(1)\\ 
X_{1}(2)\\
X_{1}(3)\\
\end{bmatrix}
-
\begin{bmatrix}
W^{0}_{8} & 0 & 0 & 0\\
0 & W^{1}_{8} & 0 & 0\\
0 & 0 & W^{2}_{8} & 0\\
0 & 0 & 0 & W^{3}_{8}
\end{bmatrix}
\begin{bmatrix}
X_{2}(0) \\ 
X_{2}(1) \\ 
X_{2}(2) \\
X_{2}(3)
\end{bmatrix}
\end{equation}

	Similarly, we can divide $x_1(n)$ into 
	\begin{align}
		x_3(n) = \cbrak{x(0), x(4)} \\
		x_4(n) = \cbrak{x(2), x(6)}
	\end{align}
	i.e.,
	\begin{align}
		\mymat{X_3(0) \\ X_3(1)} = \vec{F}_2 \mymat{x(0) \\ x(4)} \\
		\mymat{X_4(0) \\ X_4(1)} = \vec{F}_2 \mymat{x(2) \\ x(6)}
	\end{align}
	to get
	\begin{align}
		X_1(k) = X_3(k) + W_4^k X_4(k) \\
		X_1(k + 2) = X_3(k) - W_4^k X_4(k) 
	\end{align}
	for $k = 0, 1$
	\begin{equation}
\begin{bmatrix}
X_{1}(0) \\ 
X_{1}(1)\\ 
\end{bmatrix}
=
\begin{bmatrix}
X_{3}(0) \\ 
X_{3}(1)\\ 
\end{bmatrix}
+
\begin{bmatrix}
W^{0}_{4} & 0\\
0 & W^{1}_{4}
\end{bmatrix}
\begin{bmatrix}
X_{4}(0) \\ 
X_{4}(1) \\ 
\end{bmatrix}
\end{equation}
\begin{equation}
\begin{bmatrix}
X_{1}(2) \\ 
X_{1}(3)\\ 
\end{bmatrix}
=
\begin{bmatrix}
X_{3}(0) \\ 
X_{3}(1)\\ 
\end{bmatrix}
-
\begin{bmatrix}
W^{0}_{4} & 0\\
0 & W^{1}_{4}
\end{bmatrix}
\begin{bmatrix}
X_{4}(0) \\ 
X_{4}(1) \\ 
\end{bmatrix}
\end{equation}

	And on dividing $x_2(n)$ into
	\begin{align}
		x_5(n) = \cbrak{x(1), x(5)} \\
		x_6(n) = \cbrak{x(3), x(7)}
	\end{align}
	i.e.,
	\begin{align}
		\mymat{X_5(0) \\ X_5(1)} = \vec{F}_2 \mymat{x(1) \\ x(5)} \\
		\mymat{X_6(0) \\ X_6(1)} = \vec{F}_2 \mymat{x(3) \\ x(7)}
	\end{align}
	to get
	\begin{align}
		X_2(k) = X_5(k) + W_4^k X_6(k) \\
		X_2(k + 2) = X_5(k) - W_4^k X_6(k) 
	\end{align}
	for $k = 0, 1$
\begin{equation}
\begin{bmatrix}
X_{2}(0) \\ 
X_{2}(1)\\ 
\end{bmatrix}
=
\begin{bmatrix}
X_{5}(0) \\ 
X_{5}(1)\\ 
\end{bmatrix}
+
\begin{bmatrix}
W^{0}_{4} & 0\\
0 & W^{1}_{4}
\end{bmatrix}
\begin{bmatrix}
X_{6}(0) \\ 
X_{6}(1) \\ 
\end{bmatrix}
\end{equation}
\begin{equation}
	\begin{bmatrix}
	X_{2}(2) \\ 
	X_{2}(3)\\ 
	\end{bmatrix}
	=
	\begin{bmatrix}
	X_{5}(0) \\ 
	X_{5}(1)\\ 
	\end{bmatrix}
	-
	\begin{bmatrix}
	W^{0}_{4} & 0\\
	0 & W^{1}_{4}
	\end{bmatrix}
	\begin{bmatrix}
	X_{6}(0) \\ 
	X_{6}(1) \\ 
	\end{bmatrix}
	\end{equation}
		
\item For 
    \begin{align}
	    \vec{x} = \myvec{1\\2\\3\\4\\2\\1}
        \label{eq:equation1}
    \end{align}
    compte the DFT  
		using 
	    \eqref{eq:dft-mat-def}\\
		\solution Download the following Python code that plots Fig. \ref{fig-7.11}.
	\begin{lstlisting}
		wget https://github.com/gunjitmittal/EE3900/blob/main/Assignment-1/codes/7_11.py
	\end{lstlisting}
	
	Run the code by executing
	\begin{lstlisting}
		python 7.11.py
	\end{lstlisting}

	\begin{figure}[!ht]
		\centering
		\includegraphics[width=\columnwidth]{./figs/7.11.png}
		\caption{Plot of the discrete fourier transform of $\vec{x}$ using the DFT matrix}
		\label{fig-7.11}	
	\end{figure}
    \item Repeat the above exercise using the FFT
	    after zero padding $\vec{x}$.\\
		\solution Download the following Python code that plots Fig. \ref{fig-7.12}.
	\begin{lstlisting}
		wget https://github.com/gunjitmittal/EE3900/blob/main/Assignment-1/codes/7.12.py
	\end{lstlisting}
	
	Run the code by executing
	\begin{lstlisting}
		python 7.12.py
	\end{lstlisting}

	\begin{figure}[!ht]
		\centering
		\includegraphics[width=\columnwidth]{./figs/7.12.png}
		\caption{Plot of the fast fourier transform of $\vec{x}$ after zero padding}
		\label{fig-7.12}	
	\end{figure}
%	    \eqref{eq:fft-mat-def}
\item Write a C program to compute the 8-point FFT. 
\solution Download the following C codes that generate the values of $X(k)$ using $8$-point FFT
	\begin{lstlisting}
		wget https://github.com/gunjitmittal/EE3900/blob/main/Assignment-1/codes/header.h
		wget https://github.com/gunjitmittal/EE3900/blob/main/Assignment-1/codes/7.13.c
	\end{lstlisting}
	
	Compile and run the C program by executing the following
	\begin{lstlisting}
		cc -lm 7.13.c
		./a.out
	\end{lstlisting}
	
	Download the following Python code that plots Fig. \ref{fig-7.13} using the data generated by the above C code
	\begin{lstlisting}
		wget https://github.com/gunjitmittal/EE3900/blob/main/Assignment-1/codes/7.13.py
	\end{lstlisting}
	
	Run the code by executing
	\begin{lstlisting}
		python 7.13.py
	\end{lstlisting}

	\begin{figure}[!ht]
		\centering
		\includegraphics[width=\columnwidth]{./figs/7.13.png}
		\caption{Plot of $\vec{X}$ by $8$-point FFT}
		\label{fig-7.13}	
	\end{figure}
	\item Compare and determine the running time complexities of FFT/IFFT and convolution graphically
	
	\solution Download the following C codes that measure the running times of both the algorithms
	\begin{lstlisting}
		wget https://github.com/gunjitmittal/EE3900/blob/main/Assignment-1/codes/header.h
		wget https://github.com/gunjitmittal/EE3900/blob/main/Assignment-1/codes/7.14.c
	\end{lstlisting}
	
	Compile and run the C program by executing the following
	\begin{lstlisting}
		cc -lm 7.14.c
		./a.out
	\end{lstlisting}
	
	Download the following Python code that plots Fig. \ref{fig-7.14} using the running times generated by the C code and fits them to appropriate functions of the input size
	\begin{lstlisting}
		wget https://github.com/gunjitmittal/EE3900/blob/main/Assignment-1/codes/7.14.py
	\end{lstlisting}
	
	Run the code by executing
	\begin{lstlisting}
		python 7.14.py
	\end{lstlisting}

	\begin{figure}[!ht]
		\centering
		\includegraphics[width=\columnwidth]{./figs/7.14.png}
		\caption{Plot of the running times of FFT/IFFT and convolution}
		\label{fig-7.14}	
	\end{figure}
	
	From the plot, it is evident that the time complexity of FFT/IFFT is $O(n \log n)$ and that of convolution is $O(n^2)$
 \end{enumerate}
 
 
 \section{Exercises}
 Answer the following questions by looking at the python code in Problem \ref{prob:output}
	
	\begin{enumerate}[label=\thesection.\arabic*]
 \item The command
	\begin{lstlisting}
	output_signal = signal.lfilter(b, a, input_signal)
	\end{lstlisting}
	in Problem \ref{prob:output} is executed through the following difference equation
	\begin{equation}
		\label{eq:iir_filter_gen}
 		\sum _{m=0}^{M}a\brak{m}y\brak{n-m}=\sum _{k=0}^{N}b\brak{k}x\brak{n-k}
	\end{equation}
	where the input signal is $x(n)$ and the output signal is $y(n)$ with initial values all 0. Replace \textbf{signal.filtfilt} with your own routine and verify.
	
	\solution On taking the $Z$-transform on both sides of the difference equation
	\begin{align}
		\sum _{m=0}^{M}a\brak{m} z^{-m} Y(z) &= \sum _{k=0}^{N}b\brak{k} z^{-k} X(z) \\
		\implies H(z) = \frac{Y(z)}{X(z)} &= \frac{\sum _{k=0}^{N}b\brak{k} z^{-k}}{\sum _{m=0}^{M}a\brak{m} z^{-m}}
	\end{align}
	
	For obtaining the discrete Fourier transform, put $z = \j \frac{2\pi i}{I}$ where $I$ is the length of the input signal and $i = 0, 1, \ldots, I-1$
	
	Download the following Python code that does the above
	\begin{lstlisting}
wget https://github.com/gunjitmittal/EE3900/blob/main/Assignment-1/codes/7_1.py
	\end{lstlisting}
	
	Run the code by executing
	\begin{lstlisting}
		python 7_1.py
	\end{lstlisting}
	
	\item Repeat all the exercises in the previous sections for the above $a$ and $b$
	
	\solution The polynomial coefficients obtained are
	\begin{align}
		\vec{a} = \myvec{1.000 \\ -2.519 \\ 2.561 \\ -1.206 \\ 0.220} \qquad
		\vec{b} = \myvec{0.003 \\ 0.014 \\ 0.021 \\ 0.014 \\ 0.003}
	\end{align}
	
	The difference equation is then given by
	\begin{equation}
		\vec{a}^\top \vec{y} = \vec{b}^\top \vec{x} 
	\end{equation}
	
	where
	\begin{align}
		\vec{y} = \myvec{y(n) \\ y(n-1) \\ y(n-2) \\ y(n-3) \\ y(n-4)} \qquad
		\vec{x} = \myvec{x(n) \\ x(n-1) \\ x(n-2) \\ x(n-3) \\ x(n-4)}
	\end{align}
	
	We have
	\begin{equation}
		H(z) = \frac{Y(z)}{X(z)} = \frac{\sum _{k=0}^{N}b\brak{k} z^{-k}}{\sum _{m=0}^{M}a\brak{m} z^{-m}}
	\end{equation}
	
	By using partial fraction decomposition, we can write this as
	\begin{equation}
		H(z) = \sum_i \frac{r(i)}{1-p(i)z^{-1}} + \sum_j k(j)z^{-j}
	\end{equation}
	
	On taking the inverse $Z$-transform on both sides by using \eqref{eq:anun}
	\begin{align}
		H(z) &\ztrans h(n) \\
		\frac{1}{1-p(i)z^{-1}} &\ztrans \brak{p(i)}^nu(n) \\
		z^{-j} &\ztrans \delta(n-j) 
	\end{align}
	
	Thus
	\begin{align}
		h(n) = \sum_i r(i)\brak{p(i)}^nu(n) + \sum_j k(j)\delta(n-j)
	\end{align}
	
	Download the following Python code
	\begin{lstlisting}
wget https://github.com/gunjitmittal/EE3900/blob/main/Assignment-1/codes/7_2.py
	\end{lstlisting}
	
	Run the code by executing
	\begin{lstlisting}
python 7_2.py
	\end{lstlisting}
	
	The above code outputs the values of $r(i), p(i), k(i)$
	\begin{multline}
		h(n) = 
		 (0.24 - 0.71\j)(0.56 + 0.14\j)^n u(n) \\
		+ (0.24 + 0.71\j)(0.56 - 0.14\j)^n u(n) \\
		+ (-0.25 + 0.12\j)(0.70 + 0.41\j)^n u(n) \\
		+ (-0.25 - 0.12\j)(0.70 - 0.41\j)^n u(n) \\
		+ 0.016\delta(n)
	\end{multline}
	
	\begin{figure}[!ht]
		\centering
		\includegraphics[width=\columnwidth]{./figs/7.2.1.png}
		\caption{Plot of $y(n)$}
		\label{fig-7.2.1}	
	\end{figure}
	
	\begin{figure}[!ht]
		\centering
		\includegraphics[width=\columnwidth]{./figs/7.2.2.png}
		\caption{Plot of $\abs{H(e^{\j\omega})}$}
		\label{fig-7.2.2}	
	\end{figure}
	
	\begin{figure}[!ht]
		\centering
		\includegraphics[width=\columnwidth]{./figs/7.2.3.png}
		\caption{Plot of $h(n)$}
		\label{fig-7.2.3}	
	\end{figure}
	
	\item What is the sampling frequency of the input signal?
	
	\solution The sampling frequency of the input signal is \SI{44100}{\hertz} = \SI{44.1}{\kilo\hertz}
	
	\item What is the type, order and cutoff frequency of the above Butterworth filter?
	
	\solution 
	
	Type: low-pass
	
	Order: 4
	
	Cutoff frequency: \SI{4000}{\hertz} = \SI{4}{\kilo\hertz}
	
	\item Modify the code with different input parameters to get the best possible output.
	
	\solution
	
	Order: 10
	
	Cutoff frequency: \SI{3000}{\hertz} = \SI{3}{\kilo\hertz}
 \end{enumerate}
 
\end{document}   