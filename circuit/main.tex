\documentclass[journal,12pt,twocolumn]{IEEEtran}

\usepackage{enumitem}
\usepackage{amsmath}
\usepackage{amssymb}
\usepackage{gensymb}
\usepackage{graphicx}
\usepackage{txfonts}         
\usepackage{listings}
\usepackage{lstautogobble}
\usepackage{mathtools}
\usepackage{bm}
\usepackage{hyperref}
\usepackage{polynom}
\usepackage{siunitx}
\usepackage{verbatim} 
\usepackage[siunitx]{circuitikz}

\newcommand{\solution}{\noindent \textbf{Solution: }}
\providecommand{\pr}[1]{\ensuremath{\Pr\left(#1\right)}}
\providecommand{\brak}[1]{\ensuremath{\left(#1\right)}}
\providecommand{\cbrak}[1]{\ensuremath{\left\{#1\right\}}}
\providecommand{\sbrak}[1]{\ensuremath{\left[#1\right]}}
\providecommand{\mean}[1]{E\left[ #1 \right]}
\providecommand{\var}[1]{\mathrm{Var}\left[ #1 \right]}
\providecommand{\der}[1]{\mathrm{d} #1}
\providecommand{\gauss}[2]{\mathcal{N}\ensuremath{\left(#1,#2\right)}}
\providecommand{\mbf}{\mathbf}
\providecommand{\abs}[1]{\left\vert#1\right\vert}
\providecommand{\norm}[1]{\left\lVert#1\right\rVert}
\providecommand{\z}[1]{{\mathcal{Z}}\cbrak{#1}}
\providecommand{\ztrans}{\overset{\mathcal{Z}}{ \rightleftharpoons}}
\providecommand{\system}[1]{\overset{\mathcal{#1}}{ \longleftrightarrow}}
\providecommand{\parder}[2]{\frac{\partial}{\partial #2} \brak{#1}}

\let\StandardTheFigure\thefigure
\let\vec\mathbf

\numberwithin{equation}{section}
\numberwithin{figure}{section}
\renewcommand{\thefigure}{\theenumi}
\renewcommand\thesection{\arabic{section}}

\newcommand{\myvec}[1]{\ensuremath{\begin{pmatrix}#1\end{pmatrix}}}
\newcommand{\mymat}[1]{\ensuremath{\begin{bmatrix}#1\end{bmatrix}}}
\newcommand{\mydet}[1]{\ensuremath{\begin{vmatrix}#1\end{vmatrix}}}
\newcommand{\define}{\stackrel{\triangle}{=}}

\DeclareMathOperator*{\argmin}{arg\,min}
\DeclareMathOperator*{\argmax}{arg\,max}

\makeatletter
\def\pld@CF@loop#1+{%
    \ifx\relax#1\else
        \begingroup
          \pld@AccuSetX11%
          \def\pld@frac{{}{}}\let\pld@symbols\@empty\let\pld@vars\@empty
          \pld@false
          #1%
          \let\pld@temp\@empty
          \pld@AccuIfOne{}{\pld@AccuGet\pld@temp
                            \edef\pld@temp{\noexpand\pld@R\pld@temp}}%
           \pld@if \pld@Extend\pld@temp{\expandafter\pld@F\pld@frac}\fi
           \expandafter\pld@CF@loop@\pld@symbols\relax\@empty
           \expandafter\pld@CF@loop@\pld@vars\relax\@empty
           \ifx\@empty\pld@temp
               \def\pld@temp{\pld@R11}%
           \fi
          \global\let\@gtempa\pld@temp
        \endgroup
        \ifx\@empty\@gtempa\else
            \pld@ExtendPoly\pld@tempoly\@gtempa
        \fi
        \expandafter\pld@CF@loop
    \fi}
\def\pld@CMAddToTempoly{%
    \pld@AccuGet\pld@temp\edef\pld@temp{\noexpand\pld@R\pld@temp}%
    \pld@CondenseMonomials\pld@false\pld@symbols
    \ifx\pld@symbols\@empty \else
        \pld@ExtendPoly\pld@temp\pld@symbols
    \fi
    \ifx\pld@temp\@empty \else
        \pld@if
            \expandafter\pld@IfSum\expandafter{\pld@temp}%
                {\expandafter\def\expandafter\pld@temp\expandafter
                    {\expandafter\pld@F\expandafter{\pld@temp}{}}}%
                {}%
        \fi
        \pld@ExtendPoly\pld@tempoly\pld@temp
        \pld@Extend\pld@tempoly{\pld@monom}%
    \fi}
\makeatother

\lstset {
	frame=single, 
	breaklines=true,
	columns=fullflexible,
	autogobble=true
}             
                               
\title{Circuits and Transforms \\ \Large EE3900: Linear Systems and Signal Processing \\ \large Indian Institute of Technology Hyderabad}
\author{Gunjit Mittal \\ \normalsize AI21BTECH11004 \\ \vspace*{20pt} \normalsize 8 Oct 2022}

\begin{document}

	\maketitle

	\section{Definitions}
	\begin{enumerate}[label=\thesection.\arabic*,ref=\thesection.\theenumi]
	\item The unit step function is defined as
	\begin{align}
		u(t) =
		\begin{cases}
			1 & t > 0 \\
			\frac{1}{2} & t = 0 \\
			0 & t < 0
		\end{cases}
	\end{align}
		
	\item The Laplace transform of $g(t)$ is defined as 
	\begin{align}
		G(s) = \int_{-\infty}^{\infty} g(t) e^{-st}\, \der{t}
	\end{align}
	\end{enumerate}
	
	\section{Laplace Transform}
	\begin{enumerate}[label=\thesection.\arabic*.,ref=\thesection.\theenumi]
	\item In the circuit, the switch S is connected to position P for a long time so that the charge on the capacitor becomes $q_1$ \SI{}{\micro\coulomb}. Then S is switched to position Q.  After a long time, the charge on the capacitor is $q_2$ \SI{}{\micro\coulomb}
	
	\item Draw the circuit using latex-tikz
	
	\solution
	\begin{figure}[!ht]
		\centering
		\begin{circuitikz} \draw
			(0,3) to[battery1, l_=1<\volt>] (0,0) -- (8,0)
			(8,3) to[battery1, l=2<\volt>] (8,0)
			(4,0) to[C=1<\micro\farad>] (4,3)
				to[R=$2\,\Omega$] (8,3)
			(4,3) to[R, l_=$1\,\Omega$] (1,3)
				to[nos, l_=S, mirror] (0,3) node[label={left:P}]{}
			(0.7,0) -- (0.7,2.7) node[label={right:Q}]{}
			;
		\end{circuitikz}
		\caption{Circuit diagram of the circuit in question}
		\label{fig:ckt}
	\end{figure}
	
	\item Find $q_1$
	
	\solution After a long time, when steady state is achieved, a capacitor behaves like an open circuit, i.e., current passing through it is zero
	\begin{figure}[!ht]
		\centering
		\begin{circuitikz} \draw
			(0,3) to[battery1, l_=1<\volt>] (0,0) node[label={below:0}]{} 
				-- (8,0) node[label={below:0}]{}
			(8,3) node[label={above:2}]{} to[battery1, l=2<\volt>] (8,0)
			(4,3) node[label={above:$V$}] {} to[R=$2\,\Omega$] (8,3)
			(4,3) to[R, l_=$1\,\Omega$] (0,3) node[label={above:1}]{}
			(4,0) node[ground]{}
			;
		\end{circuitikz}
		\caption{Circuit diagram at steady state before flipping the switch}
	\end{figure}
	
	By Kirchoff's junction law, we get
	\begin{align}
		\frac{V - 1}{1} &+ \frac{V - 2}{2} = 0 \\
		\implies V &= \SI[parse-numbers=false]{\frac{4}{3}}{\volt} \\
		\implies q_1 &= CV = \SI[parse-numbers=false]{\frac{4}{3}}{\micro\coulomb}
	\end{align}
	
	\item Show that the Laplace transform of $u(t)$ is $\frac{1}{s}$ and find the ROC
	
	\solution The Laplace transform of $u(t)$ is given by
	\begin{align}
    		\mathcal{L}\cbrak{u(t)} &=\int_{-\infty}^{\infty}u(t)e^{-st} \,\der{t} \\
        &= \int_{0}^{\infty}e^{-st} \,\der{t} \\
        &= \lim_{R \to \infty} \frac{1 - e^{-sR}}{s}
	\end{align}
	
	This limit is finite only if $\Re(s) > 0$, which is going to be its ROC
	
	Therefore
	\begin{align}
		u(t) \system{L} \frac{1}{s} \qquad \Re(s) > 0
	\end{align}
	
	\item Show that 
	\begin{align}
		e^{-at}u(t) \system{L} \frac{1}{s+a} \qquad a > 0
	\end{align}
	and find the ROC
	
	\solution The Laplace transform of $e^{-at}u(t)$ for $a>0$ is given by
	\begin{align}
    		\mathcal{L}\cbrak{u(t)} &=\int_{-\infty}^{\infty}e^{-at}u(t)e^{-st} \,\der{t} \\
        &= \int_{0}^{\infty}e^{-(s+a)t} \,\der{t} \\
        &= \lim_{R \to \infty} \frac{1 - e^{-(s+a)R}}{s+a}
	\end{align}	
	
	This limit is finite only if $\Re(s + a) > 0$, which is going to be its ROC
	
	Therefore
	\begin{align}
		e^{-at}u(t) \system{L} \frac{1}{s+a} \qquad \Re(s) > -a
	\end{align}
	since $a$ is real
	
	\item Now consider the following resistive circuit transformed from Fig. \ref{fig:ckt}
	\begin{figure}[!ht]
		\centering
		\begin{circuitikz} \draw
			(0,3) to[battery1, l_=$V_1(s)$] (0,0) node[label={below:0}]{} 
				-- (6,0) node[label={below:0}]{}
			(6,3) node[label={above:$V_2$}]{} to[battery1, l_=$V_2(s)$] (6,0)
			(3,3) node[label={above:$V_c(s)$}] {} to[generic, l=$R_2$] (6,3)
			(3,3) to[generic, l_=$R_1$] (0,3) node[label={above:$V_1$}]{}
			(3,0) node[ground]{} to[generic, l=$\frac{1}{sC_0}$] (3,3)
			;
		\end{circuitikz}
		\caption{Circuit diagram in $s$-domain before flipping the switch}
		\label{fig:lap}
	\end{figure}
		
	where 
	\begin{align}
		u(t) \system{L} V_1(s) \\
		2u(t) \system{L} V_2(s)
	\end{align}
	Find the voltage across the capacitor $V_c(s)$
	
	\solution 
	\begin{align}
		V_1(s) &= \frac{1}{s} &&\Re(s) > 0 \\
		V_2(s) &= \frac{2}{s} &&\Re(s) > 0 
	\end{align}		
		
	By Kirchoff's junction law, we get
	\begin{align}
		&\frac{V_c - V_1}{R_1} + 	\frac{V_c - V_2}{R_2} + \frac{V_c - 0}{\frac{1}{sC_0}} = 0 \\
		\implies &V_c \brak{\frac{1}{R_1} + \frac{1}{R_2} + sC_0} = \frac{V_1}{R_1} + \frac{V_2}{R_2} \\
		\implies &V_c(s) = \frac{\frac{1}{sR_1} + \frac{2}{sR_2}}{\frac{1}{R_1} + \frac{1}{R_2} + sC_0} \\
		&\qquad = \frac{\frac{1}{R_1C_0} + \frac{2}{R_2C_0}}{s\brak{s + \frac{1}{R_1C_0} + \frac{1}{R_2C_0}}} 
	\end{align}
	
	\item Find $v_c(t)$. Plot using Python.

	\solution On performing partial fraction decomposition
	\begin{align} 
		V_c(s) &=  \frac{\frac{1}{R_1C_0} + \frac{2}{R_2C_0}}{\frac{1}{R_1C_0} + \frac{1}{R_2C_0}} \brak{\frac{1}{s} - \frac{1}{s + \frac{1}{R_1C_0} + \frac{1}{R_2C_0}}}, \Re(s) > 0
	\end{align}
	
	On taking the inverse Laplace transform, we get
	\begin{align}
	v_c(t) &= \frac{2R_1 + R_2}{R_1 + R_2} \brak{u(t) - e^{-\brak{\frac{1}{R_1} + \frac{1}{R_2}} \frac{t}{C_0}} u(t)} \\
	&= \frac{2R_1 + R_2}{R_1 + R_2} \brak{1 - e^{-\brak{\frac{1}{R_1} + \frac{1}{R_2}} \frac{t}{C_0}} }u(t)
	\end{align}
	
	Substitute the values $R_1 = \SI{1}{\ohm}, R_2 = \SI{2}{\ohm}, C_0 = \SI{1}{\micro\farad}$
	\begin{align}
		v_c(t) = \SI[parse-numbers=false]{\frac{4}{3} \brak{1 - e^{-\frac{3}{2} \times 10^6 t}} u(t)}{\volt}
	\end{align}
	
	\item Verify your result using ngspice
	
	\solution Download the following codes for simulation and plotting Fig. \ref{fig-2} respectively
	\begin{lstlisting}
		wget https://github.com/gunjitmittal/EE3900/blob/main/Circuit/codes/2.8.cir
		wget https://github.com/gunjitmittal/EE3900/blob/main/Circuit/codes/2.7.py
	\end{lstlisting}
	
	Run the codes by executing
	\begin{lstlisting}
		ngspice 2.8.cir
		python 2.7.py
	\end{lstlisting}	
	
	\begin{figure}[!ht]
		\centering
		\includegraphics[width=\columnwidth]{./figs/2.png}
		\caption{Plot of $v_c(t)$ before flipping the switch}
		\label{fig-2}	
	\end{figure}
	
	\item Obtain Fig. \ref{fig:lap} using the equivalent differential equation
	
	\solution Using Kirchoff's junction law
	\begin{align}
		\frac{v_c(t) - v_1(t)}{R_1} + \frac{v_c(t) - v_2(t)}{R_2} + \frac{\der{q}}{\der{t}} = 0
	\end{align}
	where $q(t)$ is the charge on the capacitor
	
	On taking the Laplace transform on both sides of this equation
	\begin{align}
		\frac{V_c(s) - V_1(s)}{R_1} + \frac{V_c(s) - V_2(s)}{R_2} + \brak{sQ(s) - q(0^-)} = 0
	\end{align}
	
	But $q(0^-) = 0$ and 
	\begin{align}
		q(t) &= C_0v_c(t) \\
		\implies Q(s) &= C_0V_c(s)
	\end{align}
	
	Thus
	\begin{align}
		&\frac{V_c(s) - V_1(s)}{R_1} + \frac{V_c(s) - V_2(s)}{R_2} + sC_0V_c(s) = 0 \\
		\implies &\frac{V_c(s) - V_1(s)}{R_1} + 	\frac{V_c(s) - V_2(s)}{R_2} + \frac{V_c(s) - 0}{\frac{1}{sC_0}} = 0 
	\end{align}
	
	which is the same equation as the one we obtained from Fig. \ref{fig:lap}
	
	\end{enumerate}
	
	
	\section{Initial Conditions}
	\begin{enumerate}[label=\thesection.\arabic*.,ref=\thesection.\theenumi]
	\item Find $q_2$ in Fig. \ref{fig:ckt}
	
	\solution After a long time, when steady state is achieved, a capacitor behaves like an open circuit, i.e., current passing through it is zero
	
	\begin{figure}[!ht]
		\centering
		\begin{circuitikz} \draw
			(0,3) -- (0,0) node[label={below:0}]{} 
				-- (8,0) node[label={below:0}]{}
			(8,3) node[label={above:2}]{} to[battery1, l=2<\volt>] (8,0)
			(4,3) node[label={above:$V$}] {} to[R=$2\,\Omega$] (8,3)
			(4,3) to[R, l_=$1\,\Omega$] (0,3) node[label={above:0}]{}
			(4,0) node[ground]{}
			;
		\end{circuitikz}
		\caption{Circuit diagram at steady state after flipping the switch}
	\end{figure}
	
	By Kirchoff's junction law, we get
	\begin{align}
		\frac{V - 0}{1} &+ \frac{V-2}{2} = 0 \\
		\implies V &= \SI[parse-numbers=false]{\frac{2}{3}}{\volt} \\
		\implies q_2 &= CV = \SI[parse-numbers=false]{\frac{2}{3}}{\micro\coulomb}
	\end{align}
	
	\item Draw the equivalent $s$-domain resistive circuit when S is switched to position Q.  Use variables $R_1, R_2, C_0$ for the passive elements.
Use latex-tikz

	\solution
	\begin{figure}[!ht]
		\centering
		\begin{circuitikz} \draw
			(0,3) -- (0,0) node[label={below:0}]{} 
				-- (6,0) node[label={below:0}]{}
			(6,3) node[label={above:$V_2$}]{} to[battery1, l_=$V_2(s)$] (6,0)
			(3,3) node[label={above:$V_c(s)$}] {} to[generic, l=$R_2$] (6,3)
			(3,3) to[generic, l_=$R_1$] (0,3) node[label={above:0}]{}
			(3,1) to[battery1, l=$\frac{4}{3s}$] (3,0) node[ground]{}
			(3,1) to[generic, l=$\frac{1}{sC_0}$] (3,3)
			;
		\end{circuitikz}
		\caption{Circuit diagram in $s$-domain after flipping the switch}
		\label{prob:init}
	\end{figure}
	
	The battery $\frac{4}{3s}$ corresponds to the intial potential difference of $\SI[parse-numbers=false]{\frac43}{\volt}$ across the capacitor just before switching it to Q
	
	\item Find $V_c(s)$
	
	\solution 
	By Kirchoff's junction law, we get
	\begin{align}
		&\frac{V_c - 0}{R_1} + 	\frac{V_c - V_2}{R_2} + \frac{V_c - \frac{4}{3s}}{\frac{1}{sC_0}} = 0 \\
		\implies &V_c \brak{\frac{1}{R_1} + \frac{1}{R_2} + sC_0} =  \frac{V_2}{R_2} + \frac{4}{3}C_0 \\
		\implies &V_c(s) = \frac{\frac{2}{sR_2} + \frac{4}{3}C_0}{\frac{1}{R_1} + \frac{1}{R_2} + sC_0} \\
		&\qquad = \frac{\frac{2}{R_2C_0} + \frac43 s}{s\brak{s + \frac{1}{R_1C_0} + \frac{1}{R_2C_0}}} 
	\end{align}
	
	\item Find $v_c(t)$. Plot using Python
	
	\solution On performing partial fraction decomposition
	\begin{multline}
    V_{c}(s) = \frac{4}{3}\brak{\frac{1}{s + \frac{1}{R_1C_0} + \frac{1}{R_2C_0}}} \\
    + \frac{\frac{2}{R_2C_0}}{\frac{1}{R_1C_0} +\frac{1}{R_2C_0}}\brak{\frac{1}{s} - \frac{1}{s + \frac{1}{R_1C_0} + \frac{1}{R_2C_0}}} 
	\end{multline}
	for $\Re(s) > 0$
	
	On taking the inverse Laplace transform, we get
	\begin{multline}
    v_c(t) = \frac{4}{3}e^{-\brak{\frac{1}{R_1} + \frac{1}{R_2}}\frac{t}{C_0}}u(t) \\
    + \frac{2R_1}{R_1 + R_2} \brak{u(t) - e^{-\brak{\frac{1}{R_1} + \frac{1}{R_2}}\frac{t}{C_0}}u(t) } 
	\end{multline}
	
	Substitute the values $R_1 = \SI{1}{\ohm}, R_2 = \SI{2}{\ohm}, C_0 = \SI{1}{\micro\farad}$
	\begin{align}
		v_c(t) &= \frac43 e^{-\frac{3}{2} \times 10^6 t}u(t) + \frac{2}{3}\brak{1 - e^{-\frac{3}{2} \times 10^6 t}}u(t) \\
		&= \SI[parse-numbers=false]{\frac{2}{3} \brak{1 + e^{-\frac{3}{2} \times 10^6 t}} u(t)}{\volt}
	\end{align}
	
	\item Verify your result using ngspice
	
	\solution Download the following codes for simulation and plotting Fig. \ref{fig-3} respectively
	\begin{lstlisting}
		wget https://github.com/gunjitmittal/EE3900/blob/main/Circuit/codes/3.5.cir
		wget https://github.com/gunjitmittal/EE3900/blob/main/Circuit/codes/3.4.py
	\end{lstlisting}
	
	Run the codes by executing
	\begin{lstlisting}
		ngspice 3.5.cir
		python 3.4.py
	\end{lstlisting}	
	
	\begin{figure}[!ht]
		\centering
		\includegraphics[width=\columnwidth]{./figs/3.png}
		\caption{Plot of $v_c(t)$ after flipping the switch}
		\label{fig-3}	
	\end{figure}
	
	\item Find $v_c(0^-)$, $v_c(0^+)$ and $v_c(\infty)$
	
	\solution At $t = 0^-$, the switch still hasn't been switched to Q and the circuit is in steady state
	\begin{align}
		v_c(0^-) = \SI[parse-numbers=false]{\frac43}{\volt}
	\end{align}
	
	For $t \ge 0$, we can use the above formula
	\begin{align}
		v_c(0^+) &= \lim_{t\to0^+}v_c(t) = \SI[parse-numbers=false]{\frac43}{\volt} \\
		v_c(\infty) &= \lim_{t\to\infty}v_c(t) = \SI[parse-numbers=false]{\frac23}{\volt}
	\end{align}
	
	\item Obtain Fig. \ref{prob:init} using the equivalent differential equation
	
	\solution Using Kirchoff's junction law
	\begin{align}
		\frac{v_c(t) - 0}{R_1} + \frac{v_c(t) - v_2(t)}{R_2} + \frac{\der{q}}{\der{t}} = 0
	\end{align}
	where $q(t)$ is the charge on the capacitor
	
	On taking the Laplace transform on both sides of this equation
	\begin{align}
		\frac{V_c(s) - 0}{R_1} + \frac{V_c(s) - V_2(s)}{R_2} + \brak{sQ(s) - q(0^-)} = 0
	\end{align}
	
	But $q(0^-) = \frac43 C_0$ and 
	\begin{align}
		q(t) &= C_0v_c(t) \\
		\implies Q(s) &= C_0V_c(s)
	\end{align}
	
	Thus
	\begin{align}
		&\frac{V_c(s) - 0}{R_1} + \frac{V_c(s) - V_2(s)}{R_2} + \brak{sC_0V_c(s) - \frac43 C_0} = 0 \\
		\implies &\frac{V_c(s) - 0}{R_1} + 	\frac{V_c(s) - V_2(s)}{R_2} + \frac{V_c(s) - \frac{4}{3s}}{\frac{1}{sC_0}} = 0 
	\end{align}
	
	which is the same equation as the one we obtained from Fig. \ref{prob:init}
	\end{enumerate}
	\end{document}